
\documentclass[10pt,a4paper]{article}
\usepackage{f1000_styles}

%% Default: numerical citations
% \usepackage[numbers]{natbib}

%% Uncomment this lines for superscript citations instead
% \usepackage[super]{natbib}

%% Uncomment these lines for author-year citations instead
% \usepackage[round]{natbib}
% \let\cite\citep

%% lines required to use a CSL style for references
\newlength{\cslhangindent}
\setlength{\cslhangindent}{1.5em}
\newlength{\csllabelwidth}
\setlength{\csllabelwidth}{3em}
\newlength{\cslentryspacingunit} % times entry-spacing
\setlength{\cslentryspacingunit}{\parskip}
\newenvironment{CSLReferences}[2] % #1 hanging-ident, #2 entry spacing
 {% don't indent paragraphs
  \setlength{\parindent}{0pt}
  % turn on hanging indent if param 1 is 1
  \ifodd #1
  \let\oldpar\par
  \def\par{\hangindent=\cslhangindent\oldpar}
  \fi
  % set entry spacing
  \setlength{\parskip}{#2\cslentryspacingunit}
 }%
 {}
\usepackage{calc}
\newcommand{\CSLBlock}[1]{#1\hfill\break}
\newcommand{\CSLLeftMargin}[1]{\parbox[t]{\csllabelwidth}{#1}}
\newcommand{\CSLRightInline}[1]{\parbox[t]{\linewidth - \csllabelwidth}{#1}\break}
\newcommand{\CSLIndent}[1]{\hspace{\cslhangindent}#1}

%% lines to get the code chunks working

%% lines to enable bulletpoints in a new notation style
\providecommand{\tightlist}{%
  \setlength{\itemsep}{0pt}\setlength{\parskip}{0pt}}

\begin{document}
\pagestyle{fancy}

\title{Applying a Multiverse to Population Habitat Analyses}
\author[1]{Benjamin Michael Marshall*}
\author[1]{Alexander Bradley Duthie**}
\affil[1]{Biological and Environmental Sciences, Faculty of Natural Sciences, University of Stirling, Stirling, FK9 4LA, Scotland, UK}

\affil[*]{\href{mailto:benjaminmichaelmarshall@gmail.com}{\nolinkurl{benjaminmichaelmarshall@gmail.com}}}
\affil[**]{\href{mailto:alexander.duthie@stir.ac.uk}{\nolinkurl{alexander.duthie@stir.ac.uk}}}

\maketitle
\thispagestyle{fancy}

\begin{abstract}

abc

\end{abstract}

\section*{Keywords}

Movement ecology, simulation, compana, resource selection functions, step selection function, habitat preference, habitat selection, animal movement, multiverse, research choice, researcher degrees for freedom,

\clearpage
\pagestyle{fancy}

\hypertarget{introduction}{%
\section{Introduction}\label{introduction}}

abc

\hypertarget{methods}{%
\section{Methods}\label{methods}}

\hypertarget{simulating-the-scenarios}{%
\subsection{Simulating the Scenarios}\label{simulating-the-scenarios}}

\begin{itemize}
\item
  Landscape simulation.
\item
  abmAnimalMovement settings
\end{itemize}

We used the animal movement using abmAnimalMovement v.0.1.3.0 (\protect\hyperlink{ref-abmAnimalMovement}{Marshall \& Duthie, 2022}) to simulate the movement data of an animal with a predefined (i.e., known) habitat preference.
The abmAnimalMovement package provides an agent-based approach to simulating terrestrial animal movement using raster environmental data to guie the animals decisions.
We used the NLMR v.1.1.1 package (\protect\hyperlink{ref-NLMR}{Sciaini et al., 2018}) to generate the three required resource/environmental rasters: movement resistance, foraging quality, and shelter site quality.
The abmAnimalMovement package has systems for simulating activity cycles, three separate behavioural states (differing in movement characteristics and resource prioritisation), and site fidelity.
For the purposes of this study we used one of the pre-created example pseudo-species: Badger, described in the package manuscript.
In brief, the badger is a terrestrial species occupying several shelter sites, with a 8-12 hour activity cycle with minor seasonal variation, and is subject to differing movement resistance across the landscape.

For the purposes of the analysis we simplified the landscape information into categories --akin to the sort of land-use information more frequently available to researchers of animal movement.
We focused on foraging quality because it influences the greatest amounts of movement compared to sheltering or exploratory movements.
We converted the continuous foraging quality raster into a binary, where higher quality areas (greater than 0.5) are classed as 2, and lower quality areas as 0.

We used that simulated landscape and abmAnimalMovement to simulate a population of \_\_\_, that later can be sampled from.
All individuals of this population had the same simulation settings apart from starting location.
Therefore, the variation between individuals is due to stochasticity rather than variation in the predefined habitat preference.

\hypertarget{sampling-and-analysis-options}{%
\subsection{Sampling and Analysis Options}\label{sampling-and-analysis-options}}

\begin{itemize}
\tightlist
\item
  targets construction
\end{itemize}

To manage the sampling of the population and the compounding growth of subsequent analysis decisions, we used the targets v.0.14.2 and tarchetypes v.0.7.4 R packages (\protect\hyperlink{ref-targets}{Landau, 2021a},\protect\hyperlink{ref-tarchetypes}{b}).
These packages allowed branching workflow pipeline, while keeping track of object creation thereby optimising the compute time required to explore the multiverse of analysis choice.

\hypertarget{sampling}{%
\subsubsection{Sampling}\label{sampling}}

\begin{itemize}
\item
  tracking regime
\item
  sample size
\end{itemize}

The first decision in most animal movement studies will be concerning tracking regime.
This decision is frequently dictated by more practical considerations such as anatomy and behaviour of the animal, cost of the tracking devices, and environmental factors.
Here we aimed to cover a range of tracking regimes that vary in the frequency of location fixes (
), and the total duration of tracking (
).
We created subsampled datasets based on every combination of tracking frequencies and durations, provided they would result in greater than 30 datapoints per individual.

An important component of assessing population level habitat selection is the number of individuals included in analysis.
Therefore, we randomly generated a number of samples from our population of \_\_\_ simulated individuals.
We varied these samples sizes from \_\_\_ to \_\_\_ individuals, and ran \_\_\_ repeats for each size.
A sample never mixed tracking regimes.

\hypertarget{analysis}{%
\subsubsection{Analysis}\label{analysis}}

Building on the decisions concerning tracking regime and population sampling, our multiverse expanded dramatically by exploring four primary analysis routes.
These routes included an area based approach using Compana analysis, and three step-based approaches including averaged individual step-selection models, two-step conditional regression models, and a Poisson model.

\begin{itemize}
\item
  area based: compana, area method, contour, available points, space sampling, type II/III, compana test
  \emph{adehabitatHS} v.0.3.16 (\protect\hyperlink{ref-adehabitatHS}{Calenge \& Mathieu Basille, 2023}),
  ctmm package v.0.6.1 (\protect\hyperlink{ref-ctmm}{Fleming \& Calabrese, 2023})
\item
  ssf: Model Formula (SSF or iSSF), Available Points per Step, Distribution of Step Lengths, Distribution of Turn Angles, Model Averaging Method
\end{itemize}

amt v.0.1.7 (\protect\hyperlink{ref-amt}{Signer, Fieberg \& Avgar, 2019})

\begin{itemize}
\tightlist
\item
  twoStep: Model Formula (SSF or iSSF), Available Points per Step, Distribution of Step Lengths, Distribution of Turn Angles
\end{itemize}

TwoStepCLogit v.1.2.5 (\protect\hyperlink{ref-TwoStepCLogit}{Craiu et al., 2016})

\begin{itemize}
\tightlist
\item
  poisson: Model Formula (SSF or iSSF), Available Points per Step, Distribution of Step Lengths, Distribution of Turn Angles
\end{itemize}

INLA v.23.4.24 (\protect\hyperlink{ref-rue_approximate_2009}{Rue, Martino \& Chopin, 2009}; \protect\hyperlink{ref-lindgren_explicit_2011}{Lindgren, Rue \& Lindström, 2011}; \protect\hyperlink{ref-martins_bayesian_2013}{Martins et al., 2013}; \protect\hyperlink{ref-rue_bayesian_2017}{Rue et al., 2017}; \protect\hyperlink{ref-kourounis_towards_2018}{Kourounis, Fuchs \& Schenk, 2018})

Muff, Signer \& Fieberg (\protect\hyperlink{ref-muff_accounting_2020}{2020})

\hypertarget{assessing-the-multiverse}{%
\subsection{Assessing the multiverse}\label{assessing-the-multiverse}}

\begin{itemize}
\item
  spec curves
\item
  brm models: one per each analysis method
\end{itemize}

Specification curves provide an overview of the estimates of a given range of analyses.
Tighter more steep curves suggest greater agreement between all the analysis end points.
Here we also plotted the estimates against the different decisions that results in the estimates, allowing direct comparison on how the decision impacts the variation in the estimates.

To better detect the impact of decisions, while accounting for the random variation stemming from the differences in individuals/samples, we ran a number of Bayesian Regression Models.
The Bayesian Regression Models aimed to describe how much of the deviation from a median answer could be explained by the various sampling and analysis decisions.
For each analysis route we ran a model that included tracking frequency, tracking duration, sample size, and all the corresponding analysis choices.
All continuous variables were scaled to help determine their relative importance to each other.
For the area based Compana approach the population effects included: the continuous variable contourScaled; and the categoric predictors samplingPatternst, testrandomisation, areaMethodMCP, and typeIII.
For the step-based approaches they all included: modelFormulamf.ss, stepDistgamma, turnDistvonmises.
The step-selection model approach also included: averagingMethodNaiveaverage.

\hypertarget{results}{%
\section{Results}\label{results}}

\hypertarget{specification-curves}{%
\subsection{Specification Curves}\label{specification-curves}}

(Fig. \ref{fig:specCurveArea}).

\begin{figure}
\includegraphics[width=1\linewidth]{../figures/area_specCurve} \caption{Spec curve}\label{fig:specCurveArea}
\end{figure}

(Fig. \ref{fig:specCurveSSF}).

\begin{figure}
\includegraphics[width=1\linewidth]{../figures/ssf_specCurve} \caption{Spec curve}\label{fig:specCurveSSF}
\end{figure}

(Fig. \ref{fig:specCurveTwoStep}).

\begin{figure}
\includegraphics[width=1\linewidth]{../figures/twoStep_specCurve} \caption{Spec curve}\label{fig:specCurveTwoStep}
\end{figure}

(Fig. \ref{fig:specCurvePois}).

\begin{figure}
\includegraphics[width=1\linewidth]{../figures/pois_specCurve} \caption{Spec curve}\label{fig:specCurvePois}
\end{figure}

\hypertarget{model-results}{%
\subsection{Model Results}\label{model-results}}

The conditional \emph{R\textsuperscript{2}} values differed for the three models. The Compana results model had a conditional \emph{R\textsuperscript{2}} of 0.33; whereas the SSF model returned 0.59, and the Poisson model returned 0.94.

The marginal \emph{R\textsuperscript{2}} represents the bulk of the conditional \emph{R\textsuperscript{2}} suggesting an important role for the fixed/population effects. The Compana results model had a conditional \emph{R\textsuperscript{2}} of 0.48; whereas the SSF model returned 0.51, and the Poisson model returned 0.83.

The sample size was negatively correlated with deviation from the median estimate (\(\beta\) -0.03; 95\% HDCI -1.15 - 1.8).

(Fig. \ref{fig:effectPlotArea}).

\begin{figure}
\includegraphics[width=1\linewidth]{../figures/areaBrms_effectsPlot} \caption{Beta coefs}\label{fig:effectPlotArea}
\end{figure}

(Fig. \ref{fig:effectPlotSSF}).

\begin{figure}
\includegraphics[width=1\linewidth]{../figures/ssfBrms_effectsPlot} \caption{Beta coefs}\label{fig:effectPlotSSF}
\end{figure}

(Fig. \ref{fig:effectPlotTwoStep}).

\begin{figure}
\includegraphics[width=1\linewidth]{../figures/twoStepBrms_effectsPlot} \caption{Beta coefs}\label{fig:effectPlotTwoStep}
\end{figure}

(Fig. \ref{fig:effectPlotPois}).

\begin{figure}
\includegraphics[width=1\linewidth]{../figures/poisBrms_effectsPlot} \caption{Beta coefs}\label{fig:effectPlotPois}
\end{figure}

\hypertarget{discussion}{%
\section{Discussion}\label{discussion}}

\hypertarget{limitations}{%
\subsection{Limitations}\label{limitations}}

\hypertarget{conclusions}{%
\subsection{Conclusions}\label{conclusions}}

\hypertarget{acknowledgements}{%
\section{Acknowledgements}\label{acknowledgements}}

BMM was funded by the Natural Environment Research Council (NERC) via the IAPETUS2 Doctoral Training Partnership.

\hypertarget{software-availablity}{%
\section{Software availablity}\label{software-availablity}}

In addition to packages already mentioned in the methods we also used the following.

We used \emph{R} v.4.2.2 (\protect\hyperlink{ref-base}{R Core Team, 2023}) via \emph{RStudio} v.2023.6.2.561 (\protect\hyperlink{ref-rstudio}{RStudio Team, 2022}).
We used \emph{here} v.1.0.1 (\protect\hyperlink{ref-here}{Müller, 2020}) and \emph{qs} v.0.25.5 (\protect\hyperlink{ref-qs}{Ching, 2023}) to manage directory addresses and saved objects.

We used \emph{raster} v.3.6.14 (\protect\hyperlink{ref-raster}{Hijmans, 2023}) and \emph{RandomFields} v.3.3.14 (\protect\hyperlink{ref-RandomFields}{Schlather et al., 2015}) to aid landscape raster creation alongside NLMR v.1.1.1 (\protect\hyperlink{ref-NLMR}{Sciaini et al., 2018}).

We used \emph{ggplot2} v.3.4.2 for creating figures (\protect\hyperlink{ref-ggplot2}{Wickham, 2016}), with the expansions: \emph{patchwork} v.1.1.2 (\protect\hyperlink{ref-patchwork}{Pedersen, 2022}), \emph{ggridges} v.0.5.4 (\protect\hyperlink{ref-ggridges}{Wilke, 2022}), and \emph{ggdist} v.3.2.0 (\protect\hyperlink{ref-ggdist}{Kay, 2023a}).

We used \emph{brms} v.2.19.0 (\protect\hyperlink{ref-brms}{Bürkner, 2021}) to run Bayesian models, with dianogistics generated used \emph{bayesplot} v.1.10.0 (\protect\hyperlink{ref-bayesplot}{Gabry et al., 2019}), \emph{tidybayes} v.3.0.2 (\protect\hyperlink{ref-tidybayes}{Kay, 2023b}), and \emph{performance} v.0.10.2 (\protect\hyperlink{ref-performance}{Lüdecke et al., 2021}).

We used the \emph{dplyr} v.1.0.10 (\protect\hyperlink{ref-dplyr}{Wickham et al., 2023}), \emph{tibble} v.3.1.8 (\protect\hyperlink{ref-tibble}{Müller \& Wickham, 2023}),
and \emph{stringr} v.1.5.0 (\protect\hyperlink{ref-stringr}{Wickham, 2022}) packages for data manipulation.

We used \emph{sp} v.1.5.1 (\protect\hyperlink{ref-sp}{Bivand, Pebesma \& Gomez-Rubio, 2013}), \emph{adehabitatHR} v.0.4.20 (\protect\hyperlink{ref-adehabitatHR}{Calenge \& Scott Fortmann-Roe, 2023}), \emph{move} v.4.1.12 (\protect\hyperlink{ref-move}{Kranstauber, Smolla \& Scharf, 2023}) for manipulation of spatial data and estimation of space use not otherwise mentioned in the methods.

We used rmarkdown v.2.19 (\protect\hyperlink{ref-rmarkdown2018}{Xie, Allaire \& Grolemund, 2018}; \protect\hyperlink{ref-rmarkdown2020}{Xie, Dervieux \& Riederer, 2020}; \protect\hyperlink{ref-rmarkdown2023}{Allaire et al., 2023}), bookdown v.0.33 (\protect\hyperlink{ref-bookdown2016}{Xie, 2016}, \protect\hyperlink{ref-R-bookdown}{2022}), tinytex v.0.44 (\protect\hyperlink{ref-tinytex2019}{Xie, 2019}, \protect\hyperlink{ref-tinytex2023}{2023a}), and knitr v.1.41 (\protect\hyperlink{ref-knitr2014}{Xie, 2014}, \protect\hyperlink{ref-knitr2015}{2015}, \protect\hyperlink{ref-knitr2023}{2023b}) packages to generate type-set outputs.

We generated R package citations with the aid of \emph{grateful} v.0.1.13 (\protect\hyperlink{ref-grateful}{Francisco Rodríguez-Sánchez, Connor P. Jackson \& Shaurita D. Hutchins, 2023}).

\hypertarget{data-availabilty}{%
\section{Data availabilty}\label{data-availabilty}}

\hypertarget{supplementary-material}{%
\section{Supplementary Material}\label{supplementary-material}}

\hypertarget{references}{%
\section*{References}\label{references}}
\addcontentsline{toc}{section}{References}

\hypertarget{refs}{}
\begin{CSLReferences}{1}{0}
\leavevmode\vadjust pre{\hypertarget{ref-rmarkdown2023}{}}%
Allaire J, Xie Y, Dervieux C, McPherson J, Luraschi J, Ushey K, Atkins A, Wickham H, Cheng J, Chang W, Iannone R. 2023. \emph{\href{https://github.com/rstudio/rmarkdown}{{rmarkdown}: Dynamic documents for r}}.

\leavevmode\vadjust pre{\hypertarget{ref-sp}{}}%
Bivand RS, Pebesma E, Gomez-Rubio V. 2013. \emph{\href{https://asdar-book.org/}{Applied spatial data analysis with {R}, second edition}}. Springer, NY.

\leavevmode\vadjust pre{\hypertarget{ref-brms}{}}%
Bürkner P-C. 2021. Bayesian item response modeling in {R} with {brms} and {Stan}. \emph{Journal of Statistical Software} 100:1--54. DOI: \href{https://doi.org/10.18637/jss.v100.i05}{10.18637/jss.v100.i05}.

\leavevmode\vadjust pre{\hypertarget{ref-adehabitatHS}{}}%
Calenge C, Mathieu Basille contributions from. 2023. \emph{\href{https://CRAN.R-project.org/package=adehabitatHS}{{adehabitatHS}: Analysis of habitat selection by animals}}.

\leavevmode\vadjust pre{\hypertarget{ref-adehabitatHR}{}}%
Calenge C, Scott Fortmann-Roe contributions from. 2023. \emph{\href{https://CRAN.R-project.org/package=adehabitatHR}{{adehabitatHR}: Home range estimation}}.

\leavevmode\vadjust pre{\hypertarget{ref-qs}{}}%
Ching T. 2023. \emph{\href{https://CRAN.R-project.org/package=qs}{{qs}: Quick serialization of r objects}}.

\leavevmode\vadjust pre{\hypertarget{ref-TwoStepCLogit}{}}%
Craiu RV, Duchesne T, Fortin D, Baillargeon S. 2016. \emph{\href{https://CRAN.R-project.org/package=TwoStepCLogit}{TwoStepCLogit: Conditional logistic regression: A two-step estimation method}}.

\leavevmode\vadjust pre{\hypertarget{ref-ctmm}{}}%
Fleming CH, Calabrese JM. 2023. \emph{{ctmm}: Continuous-time movement modeling}.

\leavevmode\vadjust pre{\hypertarget{ref-grateful}{}}%
Francisco Rodríguez-Sánchez, Connor P. Jackson, Shaurita D. Hutchins. 2023. \emph{\href{https://github.com/Pakillo/grateful}{{grateful}: Facilitate citation of r packages}}.

\leavevmode\vadjust pre{\hypertarget{ref-bayesplot}{}}%
Gabry J, Simpson D, Vehtari A, Betancourt M, Gelman A. 2019. Visualization in bayesian workflow. \emph{J. R. Stat. Soc. A} 182:389--402. DOI: \href{https://doi.org/10.1111/rssa.12378}{10.1111/rssa.12378}.

\leavevmode\vadjust pre{\hypertarget{ref-raster}{}}%
Hijmans RJ. 2023. \emph{\href{https://CRAN.R-project.org/package=raster}{{raster}: Geographic data analysis and modeling}}.

\leavevmode\vadjust pre{\hypertarget{ref-ggdist}{}}%
Kay M. 2023a. \emph{{ggdist}: Visualizations of distributions and uncertainty}. DOI: \href{https://doi.org/10.5281/zenodo.3879620}{10.5281/zenodo.3879620}.

\leavevmode\vadjust pre{\hypertarget{ref-tidybayes}{}}%
Kay M. 2023b. \emph{{tidybayes}: Tidy data and geoms for {Bayesian} models}. DOI: \href{https://doi.org/10.5281/zenodo.1308151}{10.5281/zenodo.1308151}.

\leavevmode\vadjust pre{\hypertarget{ref-kourounis_towards_2018}{}}%
Kourounis D, Fuchs A, Schenk O. 2018. \href{https://doi.org/10.1109/TPWRS.2017.2789187}{Towards the next generation of multiperiod optimal power flow solvers}. \emph{IEEE Transactions on Power Systems} PP:1--10.

\leavevmode\vadjust pre{\hypertarget{ref-move}{}}%
Kranstauber B, Smolla M, Scharf AK. 2023. \emph{\href{https://CRAN.R-project.org/package=move}{{move}: Visualizing and analyzing animal track data}}.

\leavevmode\vadjust pre{\hypertarget{ref-tarchetypes}{}}%
Landau WM. 2021b. \emph{Tarchetypes: Archetypes for targets}.

\leavevmode\vadjust pre{\hypertarget{ref-targets}{}}%
Landau WM. 2021a. \href{https://doi.org/10.21105/joss.02959}{The targets r package: A dynamic make-like function-oriented pipeline toolkit for reproducibility and high-performance computing}. \emph{Journal of Open Source Software} 6:2959.

\leavevmode\vadjust pre{\hypertarget{ref-lindgren_explicit_2011}{}}%
Lindgren F, Rue H, Lindström J. 2011. An explicit link between {Gaussian} fields and {Gaussian} {Markov} random fields: The stochastic partial differential equation approach (with discussion). \emph{Journal of the Royal Statistical Society B} 73:423--498.

\leavevmode\vadjust pre{\hypertarget{ref-performance}{}}%
Lüdecke D, Ben-Shachar MS, Patil I, Waggoner P, Makowski D. 2021. {performance}: An {R} package for assessment, comparison and testing of statistical models. \emph{Journal of Open Source Software} 6:3139. DOI: \href{https://doi.org/10.21105/joss.03139}{10.21105/joss.03139}.

\leavevmode\vadjust pre{\hypertarget{ref-abmAnimalMovement}{}}%
Marshall BM, Duthie AB. 2022. \href{https://0}{{abmAnimalMovement}: An r package for simulating animal movement using an agent-based model}. \emph{F1000} 0:0.

\leavevmode\vadjust pre{\hypertarget{ref-martins_bayesian_2013}{}}%
Martins TG, Simpson D, Lindgren F, Rue H. 2013. Bayesian computing with {INLA}: {N}ew features. \emph{Computational Statistics and Data Analysis} 67:68--83.

\leavevmode\vadjust pre{\hypertarget{ref-muff_accounting_2020}{}}%
Muff S, Signer J, Fieberg J. 2020. Accounting for individual-specific variation in habitat-selection studies: Efficient estimation of mixed-effects models using bayesian or frequentist computation. \emph{Journal of Animal Ecology} 89:80--92. DOI: \href{https://doi.org/10.1111/1365-2656.13087}{10.1111/1365-2656.13087}.

\leavevmode\vadjust pre{\hypertarget{ref-here}{}}%
Müller K. 2020. \emph{\href{https://CRAN.R-project.org/package=here}{{here}: A simpler way to find your files}}.

\leavevmode\vadjust pre{\hypertarget{ref-tibble}{}}%
Müller K, Wickham H. 2023. \emph{\href{https://CRAN.R-project.org/package=tibble}{{tibble}: Simple data frames}}.

\leavevmode\vadjust pre{\hypertarget{ref-patchwork}{}}%
Pedersen TL. 2022. \emph{\href{https://CRAN.R-project.org/package=patchwork}{Patchwork: The composer of plots}}.

\leavevmode\vadjust pre{\hypertarget{ref-base}{}}%
R Core Team. 2023. \emph{\href{https://www.R-project.org/}{R: A language and environment for statistical computing}}. Vienna, Austria: R Foundation for Statistical Computing.

\leavevmode\vadjust pre{\hypertarget{ref-rstudio}{}}%
RStudio Team. 2022. \emph{\href{http://www.rstudio.com/}{{RStudio}: Integrated development environment for r}}. Boston, MA: RStudio, PBC.

\leavevmode\vadjust pre{\hypertarget{ref-rue_approximate_2009}{}}%
Rue H, Martino S, Chopin N. 2009. Approximate {Bayesian} inference for latent {Gaussian} models using integrated nested {Laplace} approximations (with discussion). \emph{Journal of the Royal Statistical Society B} 71:319--392.

\leavevmode\vadjust pre{\hypertarget{ref-rue_bayesian_2017}{}}%
Rue H, Riebler AI, Sørbye SH, Illian JB, Simpson DP, Lindgren FK. 2017. \href{http://arxiv.org/abs/1604.00860}{Bayesian computing with {INLA}: {A} review}. \emph{Annual Reviews of Statistics and Its Applications} 4:395--421.

\leavevmode\vadjust pre{\hypertarget{ref-RandomFields}{}}%
Schlather M, Malinowski A, Menck PJ, Oesting M, Strokorb K. 2015. \href{https://www.jstatsoft.org/v63/i08/}{Analysis, simulation and prediction of multivariate random fields with package {RandomFields}}. \emph{Journal of Statistical Software} 63:1--25.

\leavevmode\vadjust pre{\hypertarget{ref-NLMR}{}}%
Sciaini M, Fritsch M, Scherer C, Simpkins CE. 2018. \href{https://doi.org/10.1111/2041-210X.13076}{NLMR and landscapetools: An integrated environment for simulating and modifying neutral landscape models in r}. \emph{Methods in Ecololgy and Evolution} 00:1--9.

\leavevmode\vadjust pre{\hypertarget{ref-amt}{}}%
Signer J, Fieberg J, Avgar T. 2019. Animal movement tools (amt): R package for managing tracking data and conducting habitat selection analyses. \emph{Ecology and Evolution} 9:880--890.

\leavevmode\vadjust pre{\hypertarget{ref-ggplot2}{}}%
Wickham H. 2016. \emph{\href{https://ggplot2.tidyverse.org}{ggplot2: Elegant graphics for data analysis}}. Springer-Verlag New York.

\leavevmode\vadjust pre{\hypertarget{ref-stringr}{}}%
Wickham H. 2022. \emph{\href{https://CRAN.R-project.org/package=stringr}{{stringr}: Simple, consistent wrappers for common string operations}}.

\leavevmode\vadjust pre{\hypertarget{ref-dplyr}{}}%
Wickham H, François R, Henry L, Müller K, Vaughan D. 2023. \emph{\href{https://CRAN.R-project.org/package=dplyr}{{dplyr}: A grammar of data manipulation}}.

\leavevmode\vadjust pre{\hypertarget{ref-ggridges}{}}%
Wilke CO. 2022. \emph{\href{https://CRAN.R-project.org/package=ggridges}{Ggridges: Ridgeline plots in 'ggplot2'}}.

\leavevmode\vadjust pre{\hypertarget{ref-knitr2014}{}}%
Xie Y. 2014. {knitr}: A comprehensive tool for reproducible research in {R}. In: Stodden V, Leisch F, Peng RD eds. \emph{Implementing reproducible computational research}. Chapman; Hall/CRC,.

\leavevmode\vadjust pre{\hypertarget{ref-knitr2015}{}}%
Xie Y. 2015. \emph{\href{https://yihui.org/knitr/}{Dynamic documents with {R} and knitr}}. Boca Raton, Florida: Chapman; Hall/CRC.

\leavevmode\vadjust pre{\hypertarget{ref-bookdown2016}{}}%
Xie Y. 2016. \emph{\href{https://bookdown.org/yihui/bookdown}{{bookdown}: Authoring books and technical documents with {R} markdown}}. Boca Raton, Florida: Chapman; Hall/CRC.

\leavevmode\vadjust pre{\hypertarget{ref-tinytex2019}{}}%
Xie Y. 2019. \href{https://tug.org/TUGboat/Contents/contents40-1.html}{{TinyTeX}: A lightweight, cross-platform, and easy-to-maintain LaTeX distribution based on TeX live}. \emph{TUGboat} 40:30--32.

\leavevmode\vadjust pre{\hypertarget{ref-R-bookdown}{}}%
Xie Y. 2022. \emph{\href{https://CRAN.R-project.org/package=bookdown}{Bookdown: Authoring books and technical documents with r markdown}}.

\leavevmode\vadjust pre{\hypertarget{ref-knitr2023}{}}%
Xie Y. 2023b. \emph{\href{https://yihui.org/knitr/}{{knitr}: A general-purpose package for dynamic report generation in r}}.

\leavevmode\vadjust pre{\hypertarget{ref-tinytex2023}{}}%
Xie Y. 2023a. \emph{\href{https://github.com/rstudio/tinytex}{{tinytex}: Helper functions to install and maintain TeX live, and compile LaTeX documents}}.

\leavevmode\vadjust pre{\hypertarget{ref-rmarkdown2018}{}}%
Xie Y, Allaire JJ, Grolemund G. 2018. \emph{\href{https://bookdown.org/yihui/rmarkdown}{R markdown: The definitive guide}}. Boca Raton, Florida: Chapman; Hall/CRC.

\leavevmode\vadjust pre{\hypertarget{ref-rmarkdown2020}{}}%
Xie Y, Dervieux C, Riederer E. 2020. \emph{\href{https://bookdown.org/yihui/rmarkdown-cookbook}{R markdown cookbook}}. Boca Raton, Florida: Chapman; Hall/CRC.

\end{CSLReferences}

\end{document}
